\title{表層表現中の情報に基づく文章構造の自動抽出}
\author{黒橋 禎夫\affiref{KUEE} \and 長尾 眞\affiref{KUEE}}

\headauthor{黒橋 禎夫・長尾 眞}
\headtitle{表層表現中の情報に基づく文章構造の自動抽出}
\affilabel{KUEE}{京都大学工学部 電気工学第二教室}
{Department of Electrical Engineering, Kyoto University}

\jabstract{
テキストや談話を理解するためには,まずその文章構造を理解する必要があ
る.文章構造に関する従来の多くの研究では,解析に用いられる知識の問題
に重点がおかれていた.しかし,量的/質的に十分な計算機用の知識が作成さ
れることはしばらくの間期待できない.本論文では,知識に基づく文理解と
いう処理を行なわずに,表層表現中の種々の情報を用いることにより科学技
術文の文章構造を自動的に推定する方法を示す.表層表現中の情報としては,
種々の手がかり表現,同一/同義の語/句の出現,2文間の類似性,の3つのも
のに着目した.実験の結果これらの情報を組み合わせて利用することにより
科学技術文の文章構造のかなりの部分が自動的に推定可能であることがわかっ
た.}


\jkeywords{文脈処理,文章構造,結束関係,表層表現}

\etitle{Automatic Detection of Discourse Structure \\
by Checking Surface Information in Sentences}

\eauthor{Sadao Kurohashi \affiref{KUEE} \and Makoto Nagao\affiref{KUEE}} 
\eabstract{
To understand a text or dialogue, one must track the discourse
structure. While work on discourse structure has mainly focused on
knowledge employed in the analysis, detailed knowledge with broad
coverage availability to computers is unlikely to be constructed for
the present. In this paper, we propose an automatic method for
detecting discourse structure by a variety of keys existing in the
surface information of sentences. We have considered three types of
clue information: clue expressions, occurrence of identical/synonymous
words/phrases, and similarity between two sentences. Experimental
results have shown that, in the case of scientific and technical
texts, considerable part of the discourse structure can be estimated
by incorporating the three types of clue information, without
performing sentence understanding processes by giving knowledge to
computers.}

\ekeywords{Discourse, Coherence relation, Surface information}

\begin{document}
\maketitle

\section{はじめに}

テキストや談話を理解するためには,{\bf 文章構造}の理解,すなわち各文が
他のどの文とどのような関係({\bf 結束関係})でつながっているかを知る必要が
ある.
文章構造に関する従来の多くの研究
\cite[など]{GroszAndSidner1986,Hobbs1979,Hobbs1985,ZadroznyAndJensen1991}
では,
文章構造の認識に必要となる知識,またそれらの知識に基づく推論の
問題に重点がおかれていた.
しかしそのような知識からのアプローチには次のような問題があると考えられる.
\begin{itemize}
\item 
辞書やコーパスからの知識の自動獲得,あるいは人手による知識ベース構築の
現状をみれば,量的/質的に十分な計算機用の知識が作成されることは
しばらくの間期待できない.
\item 
一方,オンラインテキストの急増にともない,文章処理の技術は
非常に重要になってきている@xcite.
そのため,現在利用可能な知識の範囲でどのような処理が可能であるかを
まず明らかにする必要がある.
\item 
現在の自然言語処理のターゲットの中心である科学技術文では,
文章構造理解の手がかりとなる情報が表層表現中に明示的に示されている
ことが多い.
科学技術の専門的内容を伝えるためにはそのように明示的表現を用いることが
必然的に必要であるといえる.
\end{itemize}

このような観点から,本論文では,表層表現中の種々の情報を用いることにより
科学技術文の文章構造を自動的に推定する方法を示す.
文章構造抽出のための重要な情報の一つは,
多くの研究者が指摘しているように「なぜなら」,「たとえば」などの
{\bf 手がかり語}である
\cite[など]{Cohen1984,GroszAndSidner1986,Reichman1985,Ono1989,Yamamoto1991}. 
しかし,それらだけで文章全体の構造を推定することは不可能であることから,
我々はさらに2つの情報を取り出すことを考えた.
そのひとつは同一/同義の語/句の出現であり,
これによって{\bf 主題連鎖}/{\bf 焦点-主題連鎖}の関係
@xciteを推定することができる.
もうひとつは2文間の類似性で,類似性の高い2文を見つけることによって
それらの間の{\bf 並列/対比}の関係を推定することができる.
これらの3つの情報を組み合わせて利用することにより
科学技術文の文章構造のかなりの部分が自動推定可能であることを示す.



{\unitlength=1mm

}

\section{文章構造のモデルと結束関係}

従来,文章構造のモデルとしてはその基本単位の結束関係
(2項関係)を再帰的に組み合わせる
ことによる木構造(文章構造木)が一般的に用いられてきた
\cite[など]{Cohen1984,Dalgren1988,GroszAndSidner1986,HallidayAndHassan1976,
Hobbs1979,Hobbs1985,LockmanAndKlappholz1980,Mann1984,PolanyiAndScha1984,
Reichman1985,ZadroznyAndJensen1991}.
しかし,何をその基本単位とするか,また基本単位間にどのような結束関係を
考えるかについては研究者ごとに独自の定義が与えられてきた.

本論文では,文章構造の自動抽出の可能性を示すことを第一義的に考え,
句点で区切られた文を基本単位とする
もっとも単純な文章構造モデルを採用する
.
一方,結束関係としてどれだけのものを考えればよいかという問題は,
対象とするテキストの種類に大きく依存する@xcite.
たとえば,物語文などでは過去の事象間の時系列の関係が中心となるが,
科学技術文や論説文などではそのような関係はほとんどみられない.
本論文では,従来の研究で扱われてきた種々の結束関係のうち,
対象とする科学技術文の構造を説明するために必要な
ものとして表[tab:CRelations]に示すものを考える
(付録[sec:text]に示した実験文の文章構造を図[fig:DSExam]に示す
).

{\unitlength=1mm

}

さらに,文章構造モデルに対して以下の仮定を行なうことにする.
\begin{quote}
{\bf 新たな文(入力文)は,それまでの文章構造木の右端の節点に対応する文
のいずれかに接続される.}
\end{quote}
これは,「新しい主題が導入された後は,古い主題に関する詳しい説明は
参照されない」ということを意味する
(図[fig:Assump]).
計算量の点で有効であり,また科学技術文の場合直観的に妥当であると
考えられたことからこの仮定を採用した

}.
なお,本論文で扱う実験テキスト(約200文)の各章(9章)については
この仮定のもとでそれぞれに適当な文章構造木を考えることが可能であった.
以下ではある入力文に対してそれが接続される文を{\bf 接続文},
また,接続文になり得る文章構造木の右端の節点に対応する文
を{\bf 接続候補文}とよぶことにする.

\section{文章構造の自動抽出}

{\unitlength=1mm

}

\subsection{概要}

前節で示したモデルに基づくと文章構造の解析は,
入力文章を前から1文づつ順に処理し,
各文(入力文)に対して適切な接続文と適切な結束関係を決定するという
問題になる.
この処理を行うために,表層表現中の次の3つの情報に着目する.
\begin{itemize}
  \item 種々の結束関係を示す手がかり表現
  \item 主題連鎖または焦点-主題連鎖関係における
同一/同義の語/句の出現
  \item 並列/対比関係にある2文の間の類似性
\end{itemize}
以降に示す方法によって,入力文と接続候補文に対してこれらの情報を自動的に抽出し,
対応する結束関係への確信度に変換することができる.
この処理によって入力文と各接続候補文との間のすべての
結束関係に対する確信度を計算し,最終的に最大の確信度をもつ
接続候補文と結束関係を選択する
(図[fig:NS_CS]). 

文章構造には初期状態として{\bf 初期節点}(文には対応しない)を与え,
この初期節点と入力文の間の{\bf 初期化}関係に一定の確信度を与えておく
(入力文と初期節点の組については上で述べた情報の抽出処理は行なわない).
いずれの接続候補文に対してもこの値よりも大きな確信度を持つ結束関係が
存在しない場合には,その入力文は初期節点に接続されるとする.
これはその入力文が新しい段落の始まりの文であるような場合に対応する.

以下,各情報に対してそれらを抽出し結束関係への確信度に変換する方法を
説明する.

{\unitlength=1mm

}

\subsection{手がかり表現の抽出}

種々の結束関係を示す手がかり表現を取り出しその関係への確信度を得るために,
ヒューリスティック・ルールを用意した.
ルールは以下のものからなる.
\begin{itemize}
\item {\bf ルールの適用条件} :
  \begin{itemize}
  \item {\bf ルールの適用範囲} 
    (どれだけ離れた接続候補文までルールを適用するか)
  \item {\bf 接続候補文とその接続文との結束関係}

  \item {\bf 接続候補文の依存構造のパターン}
  \item {\bf 入力文の依存構造のパターン}
  \end{itemize}
\item {\bf 対応する結束関係と確信度}
\end{itemize}
接続候補文と入力文のパターンはそれぞれの文の依存構造解析結果に対して
適用される@xcite.
この処理は依存構造木に対する柔軟なパターン照合機能を用いて実現した.
そこでは,依存構造木とその構成要素である文節(単語の並び)に対するパターンが,
正規表現,論理和,論理積,否定などによって指定できる
@xcite.
ルールは各接続候補文と入力文の組に対して適用され,条件部が
満たされれば対応する結束関係に指定された確信度の得点が与えられる
(複数のルールがマッチした場合確信度の得点は加算されていく).

ルールの一例を表[tab:HR]に示す
(すべてのルールを付録[sec:rule]に示す).
たとえば,ルールa(表[tab:HR])は
入力文が「なぜなら」で始まる場合,その入力文と直前の接続候補文
の間の理由関係に得点を与える
(ルールの適用範囲が1であるので,直前の接続候補文との間の
関係に対してのみ得点が与えられる).
ルールb(表[tab:HR])の場合は,条件部で同一の語の出現を指定しており,
直前の接続候補文だけでなく他の接続候補文に対しても適用される.
ルールc(表[tab:HR])では,条件として
「接続候補文とその接続文との結束関係」を指定している.
このルールは,
「例提示関係によって
具体例が導入されれば,次にその例の説明が続く場合がある」こと
を表現している.
ルールd(表[tab:HR])は
時制の変化を手がかりとして文章構造の区切れを検出するためのルールで,
連続する2文の時制が現在から過去に移行する場合,その間の全ての
結束関係の確信度を指定された値だけ減少させる.
このペナルティの値によって入力文が直前の文以外の接続候補文へ接続される
ことが優先される.

{\unitlength=1mm

}

{\unitlength=1mm

}

\subsection{語連鎖の抽出}

一般に文は主題を示す部分(主題部分)とそれ以外の部分(非主題部分)に分ける
ことができる.
2つの文が同じ主題について述べられている場合,それら2文は
主題連鎖関係にあるとする.
この関係は2つの文の主題部分に
同一/同義の語/句(以下これを{\bf 語連鎖}とよぶ)が現れていることで発見できる.
一方,ある文の主題以外の要素がその後の文の主題となるような結束関係を
焦点-主題連鎖関係とよぶことにする
(この時,後の文で主題となる要素は前の文において焦点要素であると
考えられるため).
この関係は2文間の主題部分から非主題部分への語連鎖を調べることで発見できる.

しかし,多くの場合一つの入力文に対して各接続候補文との種々の関係を支持する
複数の情報が存在する.
そのため,単に語連鎖を発見するだけでなく,その連鎖の強さに応じて
主題連鎖関係あるいは焦点-主題連鎖関係に確信度を与えることが必要となる.
そこで,主題部分,非主題部分の各語に対して,文中での重要度に応じた
得点を与え,また同一/同義の語/句の照合に対してもその一致度に
応じた得点を定義した.
その上で,連鎖する2語/句のそれぞれの文での重要度の得点とその一致度の
得点の総和を語連鎖の得点とし,これを
主題連鎖関係あるいは焦点-主題連鎖関係に確信度として与えるということを行った
(図[fig:TF]).

これらの処理は,依存構造に対するパターンと得点からなるルールを適用
するという形で実現した(ルールの一例を表[tab:TDCRule]に示す).
ルールa,b(表[tab:TDCRule])は主題を示す助詞「は」を伴う語とその修飾語/句に,
ルールc,d(表[tab:TDCRule])は条件節内の語に,主題部分としての得点を与える.
ここでは,主題部分の中でもっとも重要であると考えられる,助詞「は」を直接
ともなう語に最大の得点を与えている.
ルールe(表[tab:TDCRule])は「〜Aがある」という文の
Aに対して非主題部分の要素として高い得点を与える.
このような文体ではAが重要な新情報であり,
この語が以降の文で主題として取り上げられる
(焦点-主題連鎖関係が存在する) 場合が多い.
ルールf,g,h(表[tab:TDCRule])は語/句の一致に対して得点を与えるルールである.
ここでは「XのY」のような句の一致(ルールg)に対しては
単なる語の一致(ルールf)よりも高い得点を与える.

語連鎖の検出における最大の問題は,著者が,単に同一の語/句を繰り返すのではなく,
微妙に異なった表現によってそのような連鎖を示す傾向にあるという問題である.
シソーラスの利用,ルールh(表[tab:TDCRule])などによってそのような表現の一部は検出
可能であるが,検出できない微妙な表現は多数存在する.
それらの取扱いについては本論文の範囲外とした.

\subsection{2文間の類似度の計算}

並列/対比の結束関係にある2文の間には
ある種の類似性が認められる.
しかしそれらは文全体についての類似性であるため,これまでに示したような
文中の比較的狭い部分を調べるルールを適用することでは検出できない.
そこで,1文内の並列構造の範囲を調べるために開発した
ダイナミックプログラミングによる類似性検出の方法
@xcite
を拡張するということを行った.

この方法では任意の長さの文節列間の類似度を計算することが可能である.
そこでは,まず文節間の類似度を,
品詞の一致,語の一致,シソーラス辞書中での語の近さなどによって計算し,
その上でそれらの文節間の類似度を組み合わせることによって文節列間の類似度を
計算する.
この手法は並列構造の構成要素を決定するために1文内の句/節間の
類似度を計算するものとして開発した.
これを2文間(入力文と接続候補文)の類似度を計算するように拡張することは,
その2文を連結しそれらが仮想的に並列構造を構成すると見なすことによって
簡単に実現することができる.
このことにより,その仮想的並列構造の構成要素である2文間の類似度をまったく
同じ枠組みで計算することができるからである.
最終的には,この方法によってえられた類似度を2文の長さの和によって正規化した
値を,それらの間の並列関係と対比関係の確信度に加算する.


\section{実験と考察}

\subsection{実験方法と結果}

実験には科学雑誌サイエンスのテキスト,
「科学技術のためのコンピューター」(Vol.17,No.12)を用いた
(付録[sec:text]にその一部を示す).
実験文はあらかじめ章ごとに分割し(全9章,1章は平均24文),
文章構造の推定は章単位で行なった.

実験は以下の手順で行なった.
まずはじめの3章に対して.主題部分/非主題部分の重要度のルール,
語/句の照合ルール,手がかり表現のルールを作成し,できるだけ
正しい文章構造が求まるようにそれらの得点を人手で調整した
.
2文間の類似度の計算については,単に並列構造検出の
システムを用いただけで,得点付けの調整/変更は行なわなかった.
次に,残りの6章に対して手がかり表現のルールだけを追加し,
その新しいルールセットによって残り6章に対する解析実験を行なった.
これらの実験結果を表[tab:Experiment]に示す.
ここでは,各入力文に対して正しい接続文と正しい結束関係,
また結束関係が主題連鎖または焦点-主題連鎖の場合はその正しい語連鎖が求まった
場合を正解とした.
結束関係ごとの解析結果の集計では,解析失敗のものについては
その正しい結束関係の欄に分類した.
なお,残りの6章に対するルールを追加した新しいルールセットでもとの3章を
解析したところ,解析結果は全く同じであった.


これらの結果から,科学技術文の場合,
表層表現中の情報をうまく取り出すことができれば
その文章構造のかなりの部分が自動的に推定可能であることがわかる.
手がかり表現についての汎用性のあるルールセットを用意するためには,
かなりの規模のテキストを対象としたルール作成作業が必要であると思われる.
しかし,手がかり表現に関するルールの多くは
排他的なものとして記述することができるので,
本実験においてそうであったように,
新しく追加したルールがもとのルールと競合して副作用を起こすということは
非常に少ないと予想される.

\subsection{解析例と考察}

まず,文章構造推定の経緯の具体例を示す.
付録のS1-11からS1-20までの文章は以下に示す情報によって
図[fig:DSExam]-aに示す構造に変換された.
\begin{description}

\item[S1-11 --初期化@xmath0 初期節点] --- 
S1-10との関係に対するペナルティ(付録[sec:rule]ルール1).
他の接続候補文との間に大きな確信度をもつ結束関係がないため,
初期節点に接続される.

\item[S1-12 --変化@xmath1 S1-11] --- 
過去から現在への時制の変化と,手がかり語「しかし」(ルール40).

\item[S1-13 --焦点-主題連鎖@xmath2 S1-12] --- 
「合成による分析」と「合成法」の連鎖.

\item[S1-14 --理由@xmath3 S1-13] --- 
手がかり表現「からである」(ルール34).

\item[S1-15 --詳細化@xmath4 S1-14] --- 
手がかり表現「わけである」(ルール20).

\item[S1-16 --例提示@xmath5 S1-13] --- 
手がかり表現「〜の例として」(ルール41).

\item[S1-17 --詳細化@xmath6 S1-16] --- 
直前の例提示関係(ルール26).

\item[S1-18 --原因-結果@xmath7 S1-17] --- 
手がかり表現「その結果は」(ルール36).

\item[S1-19 --主題連鎖@xmath8 S1-13] --- 
「合成法」の連鎖.

\item[S1-20 --変化@xmath9 S1-19] --- S1-12と同様.

\end{description}

また付録のS4-1からS4-7までの文章は
以下の手順で図[fig:DSExam]-bに示す構造に変換された.
\begin{description}

\item[S4-2 --質問-応答@xmath10 S4-1] --- 
手がかり表現「〜か」(ルール43).

\item[S4-3 --焦点-主題連鎖@xmath11 S4-2] --- 
「連星」の連鎖.

\item[S4-4 --焦点-主題連鎖@xmath12 S4-3] --- 
「温めることがある」と「この過程」の連鎖(ルール19).

\item[S4-5 --焦点-主題連鎖@xmath13 S4-4] ---
「核融合」の連鎖.

\item[S4-6 --並列@xmath14 S4-3] ---
s4-3との類似度と手がかり表現「また」(ルール5).

\item[S4-7 --並列@xmath15 S4-6] ---
類似度による得点が高いため誤ってS4-6との並列関係が推定された.
正解はS4-6との焦点-主題関係である.

\end{description}

接続詞「しかし」は,S1-12,S1-20のように
変化関係を示す手がかり語であるだけでなく,
S3-4のように対比関係を示す場合もある.
この区別は,この手がかり語と他の情報を組み合わせて調べることによって
可能となる.
すなわち,S1-12,S1-20の場合は過去から現在への時制の変化を見ることによって
変化関係を推定することができる.
一方,S3-3とS3-4の間には高い類似度(得点23)があるため,
これと手がかり語「しかし」によって対比関係が推定できる
(これに対して,S1-11とS1-12の類似度は0,S1-19とS1-20の類似度は3である). 

連続する2文間の結束関係だけでなく,離れた2文間の関係も,
種々の情報によって正しく推定することが可能である.
例えば,S1-16,S1-12間の例提示関係は
表層表現中の手がかり語によって,
S1-19,S1-13間の主題連鎖関係は語連鎖によって,
またS4-6とS4-3の並列関係は2文間の類似度によってそれぞれ正しく推定
されている.

S4-7については正しい結束関係が推定できなかった,
ここではS4-6の「温度が上昇する」ことによって生じる「熱」がS4-7で
主題となっている.
すなわち,推論を介した焦点-主題連鎖関係が存在している.
このように結束関係の推定に推論を必要とするような問題は
本論文では対象としなかった.

このような問題を含めて,実験テキストの解析誤りの原因は次のように分類できる.
なお,学習サンプル,テストサンプルともに解析誤りの原因の種類は同じもので
あった.

\begin{enumerate}
\item 語連鎖の抽出について \\
3.3節でも述べたように,主題連鎖/焦点-主題連鎖関係を示す語連鎖には
シソーラスやある種のパターン(表[tab:TDCRule]のルールhなど)を用いる
だけでは検出不可能なものがある.
上記のS4-7はその一例である.
また逆に,主題連鎖/焦点-主題連鎖関係を示すものではない同一語句の出現に
対して得点が与えられ,それが他の正しい関係の推定の妨げとなる場合がある.
\item 2文間の類似度の計算について \\
たとえば,
\begin{quote}
「銀河の中には、互いに近接していて、橋がかかっているように見えるものがある。
しかし大多数の銀河は、ほぼ対称的な渦巻き形、あるいは単純な円形または楕円形をしている。」
\end{quote}
という対比関係の2文では,同一/同義の単語も少なく構造(品詞の並び)的にも
似ていないため我々の方法では高い類似度が与えられない.
このような場合並列/対比関係が正しく推定されないということになる.
また逆に,並列/対比関係にない2文でも同一の語を多数含んでいるような場合
高い類似度が与えられ,誤ってそれら2文間の並列/対比関係が推定されてしまう
場合がある.
\item 詳細化関係について \\
詳細化関係ではたとえば次の例のように手がかり表現といえるものが
存在しない場合がある.
\begin{quote}
「こうした終局的な型は、初期条件に左右される。
遭遇時の速度や傾きの角度を少し変えるだけで、複雑な3体の動きは大幅に変わる。」
\end{quote}
このような2文間の詳細化関係は本手法では正しく推定できない.
\end{enumerate}
表[tab:Experiment]に示すとおり解析誤りの大部分は
並列/対比関係,主題連鎖/焦点-主題連鎖関係についての誤りで,これらの多くは
上の1,2の原因が組み合わさって生じたものである.
たとえば,微妙な表現の語連鎖が抽出できない場合に,別の不適当な文との間の
高い類似度によって並列/対比関係が推定されてしまったというような場合である.

なお,本論文では同一著者のテキストを学習サンプル,テストサンプルとして
実験を行なった.
種々のテキストに対して本手法の有効性を検証することは今後の課題である.
しかし,科学技術文の本質的目的がその内容を正確に伝えることであるという
ことを考えれば,著者・テキストによって文体に差があるとしても,それは
本手法の精度に大きな影響を与えるほどのものではないと考えられる.

\section{結論}

本論文では,手がかり表現,語連鎖,文間の類似性,という
表層表現中の3つ情報に基づいて文章構造を自動的に推定する手法を示した.
科学技術文の場合,
これら3つの表層的情報を組み合わせて利用することにより,
知識に基づく文理解という処理を行なわなくても,
文章構造のかなりの部分が推定できることがわかった.
微妙な表現による語連鎖の扱い,手がかり表現についての汎用性のある
ルールセットの作成などの問題を克服すれば,
大規模なテキストに対して文単位でなく文章単位の処理を実現することが
可能となるだろう.


\end{document}