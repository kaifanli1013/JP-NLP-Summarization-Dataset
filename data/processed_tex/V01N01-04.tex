\title{}
\author{}
\jkeywords{}

\etitle{A System for Finding Translation Patterns\\
  by Comparing an MT Result and Its Correction}
\eauthor{Hideo Watanabe\affiref{NihonIBM}}

\headauthor{Hideo Watanabe}
\headtitle{\scriptsize\bf
  \shortstack[l]{A System for Finding Translation Patterns\\
  by Comparing an MT Result and Its Correction}}

\affilabel{NihonIBM}
{IBM Research, Tokyo Research Laboratory (E-mail: watanabe@trl.ibm.co.jp)}
{IBM Research, Tokyo Research Laboratory (E-mail: watanabe@trl.ibm.co.jp)} 

\eabstract{When the example-based approach is used for machine
  translation, it is important to collect a large volume of
  translation patterns, because most systems use as a translation
  example a pair of parsed structures in the source and target
  languages. Such parsed translation pairs are hard to collect. This
  paper describes a system for finding parsed translation pairs (or
  translation patterns) that are valid for the translation pattern
  base by comparing wrong translations and corresponding correct
  translations.}
 
\ekeywords{Machine Translation, Example-based Approach, 
  Translation Pattern Extraction}

\begin{document}

\maketitle

\section{Introduction}


The example-based approach (EBA), an emerging technology proposed by Nagao
@xcite, has been extensively used in many areas of machine
translation @xcite.
A typical EBA transfer system uses a large volume of translation examples, or
translation patterns, each of which is a pair of parsed structures in two
languages. Collecting a large volume of such patterns is a difficult and
time-consuming task, because it is mostly done manually. A method
for collecting such patterns automatically is therefore needed.

In considering the human task of acquiring translation patterns, we noticed
that there are at least two different approaches: one is to find translation
patterns by viewing many translation examples, and another is to find
translation patterns by comparing a wrong translation and its correction. 
Several studies @xcite have
followed the former approach. On the other hand, we have developed a system
called \TranPet\ that follows the latter; that is, one that can identify new
translation patterns by comparing an MT result and its correction.  In
\figref{fig:introxmp}, (a) is the dependency structure of an input Japanese
sentence, (r1), (r2), and (r3) are given translation patterns, and (b) is the
English dependency structure produced by these patterns.  Clearly, (b) is not
a correct translation of (a), whereas (c) is. \TranPet\ compares the current
output (b) and a correct output (c), and finds a translation pattern (r4).
\begin{figure*}[tb]
\begin{center}
\unitlength 1mm
\begin{picture}(100,100)
\put(5,5){\framebox(90,90){xmpintro.ps}}
\end{picture}
\end{center}
\caption{Example of extracting a new translation pattern}

\end{figure*}

Given an input string \Ss, let \St\ be an output string produced by a
translation system @xmath0, and let \Sc\ be a correct translation of \Ss. The
purpose of the proposed method is to obtain translation patterns that produce
the correct translation string \Sc. We assume the following:
\begin{itemize}
\item @xmath1 is an example-based transfer system that can use 
translation patterns directly.
\item Parsing is error-free; that is, the parsed structures
of the source and target languages are correct.
\item The translation pattern base includes some translation patterns.
\end{itemize}
The second condition might sound rather severe. However it does not mean that parsing
must be done entirely by a parsing program, but rather that human
supervision is allowed.  Thus, we can break down the problem of obtaining
translation patterns into the following two sub-problems:
\begin{itemize}
\item Making a mapping \Mc\ from \Ds\ to \Dc, 
\item Finding translation patterns by comparing
@xmath2 with @xmath3
\end{itemize}
where \Ds\ is a dependency structure of an input string \Ss, \Dt\ is a
dependency structure produced by @xmath4, \Dc\ is a dependency structure of
\Sc, and \Mt\ is a mapping from \Ds\ to \Dt\ easily extracted from a set of
translation patterns \Tp\ used in constructing \Dt\ by means of @xmath5.

An overview of the system \TranPet\ is given in the next section, and how an
example-based transfer system works is described briefly in
\secref{sec:ebtrn}.  A method for finding mappings is given in
\secref{sec:mapping}, and a method for finding translation patterns in
\secref{sec:findpat}.
\secref{sec:example} gives an example of the system, and \secref{sec:discuss}
discusses related work. This paper closes with some concluding remarks.

\section{System Overview}
 

The system flow of \TranPet\ is shown in \figref{fig:sysflow}.  An input
Japanese sentence \Ss\ is translated into an English sentence \St\ by a
Japanese-to-English machine translation system called {\bf JETS},\footnote
{JETS is a Japanese-to-English translation system that IBM Tokyo Research Lab
has been developping for over 10 years.} which uses an example-based transfer
system called {\bf SimTran} @xcite.
\begin{figure*}[tb]
\begin{center}
\unitlength 1mm
\begin{picture}(100,120)
\put(5,5){\framebox(90,110){sysflow.ps}}
\end{picture}
\end{center}
\caption{Flow of the system}

\end{figure*}
If the translation is wrong, a user gives the system a correct translation 
\Sc\, which is then processed by a parser for the target language to give
a dependency structure \Dc. Subsequently, the correspondence (mapping data)
between \Ds\ and \Dc\ is determined by the Mapper. Finally, the Pattern
Extractor finds relevant translation patterns by comparing \Dc\ and \Dt\
with \Ds, @xmath6 translation patterns used when creating \Dt, and mapping data.

\TranPet\ is an interactive system, which requires human interaction in
all the sub-tasks, because these tasks (parsing, finding mappings, and finding
translation patterns) are not error-free.

\section{Example-Based Transfer System}
 

This section explains briefly how the example-based transfer system \SimTran\
works.

A dependency structure used in \SimTran\ is expressed as a rooted labeled
directed acyclic graph (rldag for short) indicating a graph that has only one
root node, and has labeled nodes and arcs, directed arcs, and no cyclic arc
path.  Each node consists of several features, and each arc is labeled
according to its grammatical relation.

A translation pattern @xmath7 is expressed as the following 3-tuple:
\begin{tabbing}
xxxxx\=\+\kill
@xmath8
\end{tabbing}
where @xmath9 is a dependency structure of the source language and @xmath10 is a
dependency structure of the target language. Further, @xmath11 is the following set
of mappings:
\begin{tabbing}
xxxxx\=\+\kill
@xmath12\ML, \Mup, \Mdw@xmath13
\end{tabbing}
where \ML\ is a {\em lexical mapping} from a subset of @xmath14 nodes to a subset
of @xmath15 nodes, which indicates a @xmath16 node that is a word used to translate
a @xmath17 node, and \Mup\ and \Mdw\ are structural mappings from a subset of
@xmath18 nodes to a subset of @xmath19 nodes, which indicate a connection point
(node) in @xmath20 for a @xmath21 node, on which some @xmath22s are merged; \Mup\ is
called a {\em upward mapping} and \Mdw\ is called a {\em downward mapping}.
\begin{figure*}
\begin{center}
\unitlength 1mm
\begin{picture}(100,40)
\put(5,5){\framebox(90,30){mapping.ps}}
\end{picture}
\end{center}
\caption{Example of a translation pattern}

\end{figure*}
\figref{fig:mapping} shows an example of a translation pattern according to 
which the Japanese sentence ``kanojo ha kami ga nagai'' can be translated as
the English sentence ``she has long hair.'' In this figure, dotted lines
represent correspondences between words.  The Japanese word ``nagai''
(``long'') has two corresponding nodes ``have'' and ``long,'' as an upward
mapping and a downward mapping, respectively. Suppose that ``nagai'' has the
modifiers ``totemo'' (``very'') and ``itsumo'' (``always''); then the English
word ``very'' used to translate ``totemo'' should be related to ``long,'' but
the word ``always'' used to translate ``itsumo'' should be related to
``have.''  This is one of the reasons why upward and downward mappings are
needed.

\SimTran\ runs as follows:
\begin{enumerate}
\item Gather translation patterns all of whose @xmath23s cover
the input dependency structure completely.
\item Combine all the @xmath24s of the selected translation patterns so that
the @xmath25 nodes whose corresponding @xmath26 node is the same.
\end{enumerate}
To find a corresponding @xmath27 node for a @xmath28 node,
the above-mentioned structural mappings are used. 

Even though an example-based transfer system does not use structural mappings,
at least it has a lexical mapping. Therefore, in such systems, a lexical
mapping is used instead of structural mappings to find a corresponding @xmath29
node. For such example-based transfer systems, descriptions on structural
mappings can be ignored.

\section{Mapper}
 

This section describes how to find mappings between two dependency structures
mentioned in the previous section.

Utsuro et al. @xcite and Kaji et al. @xcite proposed methods
for finding the correspondences between translation pairs.  The former method
finds a one-to-one correspondence between two feature structures of
translation equivalents, and the latter method finds a one-to-one
correspondence between two phrase structures. This one-to-one mapping
corresponds to lexical mapping in \SimTran. Since finding structural mapping
can reduce the imcompleteness of lexical mapping, the procedure for finding
lexical mapping is simpler than Utsuro's method.

\begin{figure*}
\raggedright
\small
{\bf procedure} lexical-mapping(\Ds,\Dt) \bfbegin\ \\
\hspace{5mm} @xmath30 @xmath31 \aline{al:ba} \\
\hspace{5mm} @xmath32 nodes of \Dt\ \aline{al:bc} \\
\hspace{5mm} \bffor\ each node @xmath33 in \Ds\ \bfbegin\ \aline{al:bd} \\
\hspace{10mm} @xmath34 \aline{al:be} \\
\hspace{10mm} \bfif\ @xmath35 \bfthen\ 
add @xmath36 to @xmath37 \aline{al:bg} \\
\hspace{10mm} \bfelseif\ @xmath38 \bfthen\ 
add @xmath39 to @xmath40 \aline{al:bh} \\
\hspace{5mm} \bfend\ \aline{al:bdend} \\
\hspace{5mm} \bffor\ each @xmath41 in @xmath42 \bfbegin\ \aline{al:bi} \\
\hspace{10mm} {\bf loop begin} \aline{al:bj} \\
\hspace{15mm} @xmath43 \aline{al:bk} \\
\hspace{15mm} \bfif\ @xmath44 \bfthen\ break \aline{al:bl} \\
\hspace{15mm} @xmath45 such that @xmath46
\aline{al:bm} \\
\hspace{15mm} @xmath47
\aline{al:bn} \\
\hspace{15mm} \bfif\ @xmath48 \bfthen\ \bfbegin\ \aline{al:bo} \\
\hspace{20mm} add @xmath49 to @xmath50 \aline{al:bp} \\
\hspace{20mm} break \aline{al:bq} \\
\hspace{15mm} \bfend\ \aline{al:boend} \\
\hspace{10mm} \bfend\ \aline{al:bjend} \\
\hspace{5mm} \bfend\ \aline{al:biend} \\
\hspace{5mm} return @xmath51 \\
\bfend
\normalsize
\caption{Algorithm for finding a lexical mapping}
 
\end{figure*}
\begin{figure*}
\raggedright
\small
{\bf procedure} mapping(\Ds,\Dt) \bfbegin\ \\
\hspace{5mm} @xmath52 lexical-mapping(\Ds,\Dt)
\aline{al:aa}\\
\hspace{5mm} @xmath53 \aline{al:ab} \\
\hspace{5mm} @xmath54 \aline{al:ac} \\
\hspace{5mm} \bffor\ any node pair @xmath55 and @xmath56 in \Ds\ such that @xmath57 is an
ancestor node of @xmath58 \bfbegin \aline{al:ad} \\
\hspace{10mm} @xmath59 \aline{al:ae}\\
\hspace{10mm} @xmath60 \aline{al:af}\\
\hspace{10mm} \bfif\ @xmath61 is an ancestor node of @xmath62 \bfthen \aline{al:ag}\\
\hspace{15mm} remove @xmath63 from @xmath64 \aline{al:ah}\\
\hspace{15mm} @xmath65 \aline{al:ahh}\\
\hspace{15mm} \bffor\ each @xmath66 in ancestors of @xmath67 in \Dt\ \bfbegin
\aline{al:ai}\\  
\hspace{20mm} \bfif\ @xmath68 is a root node, or a parent node of @xmath69 or @xmath70 is related by
@xmath71 or @xmath72 \bfthen\ \bfbegin\ \aline{al:aj}\\
\hspace{30mm} @xmath73 \aline{al:ak}\\
\hspace{30mm} break \aline{al:al}\\
\hspace{20mm} \bfend\ \aline{al:am}\\
\hspace{15mm} \bfend\ \aline{al:an}\\
\hspace{15mm} add @xmath74 to @xmath75 \aline{al:ao}\\
\hspace{10mm} \bfend\ \aline{al:ap}\\
\hspace{10mm} @xmath76 \aline{al:aq}\\
\hspace{10mm} @xmath77 \aline{al:ar}\\
\hspace{10mm} \bfif\ @xmath78 \bfthen\ \bfbegin \aline{al:as}\\
\hspace{15mm} remove @xmath79 from @xmath80 \aline{al:at}\\
\hspace{15mm} add @xmath81 to @xmath82 \aline{al:au}\\
\hspace{10mm} \bfend \aline{al:av}\\
\hspace{5mm} \bfend\ \aline{al:awa}\\
\hspace{5mm} return @xmath83 \aline{al:aw} \\
\bfend\
\normalsize
\caption{Algorithm for finding a structural mapping}

\end{figure*}
\begin{figure*}
\begin{center}
\unitlength 1mm
\begin{picture}(100,50)
\put(5,5){\framebox(90,40){mappingxmp.ps}}
\end{picture}
\end{center}
\caption{Example of finding mapping and phrasal correspondence}

\end{figure*}
 
Our algorithm is shown in \figref{alg:mapping1} and \figref{alg:mapping2}. 
Briefly, this algorithm first finds a lexical mapping for a given translation
pair (line [al:aa]), and then finds structural mappings (lines
[al:ad]-[al:aw]). The lexical mapping is found as follows: First, for
each node, possible translation nodes are checked by using a source-to-target
dictionary (line [al:be]). If the translation node is uniquely determined,
then a pair of the source node and its translation node is added to the \ML\
(line [al:bg]). For each node that has multiple possible translation
nodes, a check is made to establish whether the translation node of one of its
neighbor nodes has already been determined, and if so, the translation node
nearest to that of the neighbor node is used (lines
[al:bi]-[al:biend]). The function @xmath84
returns a neighbor node of @xmath85 in @xmath86 which is not visited with regard to
@xmath87, and the function @xmath88 returns a node out
of @xmath89 which is the most neighbor to @xmath90 in @xmath91.  After that,
structural mappings are found as follows: A search is made for two
correspondences @xmath92 and @xmath93 causing head switching such that
@xmath94 is an ancestor of @xmath95 while @xmath96 is a descendant of @xmath97 (lines
[al:ad]-[al:ag]). If such node pairs are found, the algorithm searches
ancestors of @xmath98 for a node @xmath99 such that (1) @xmath100 is a root node, or (2)
a parent node of @xmath101 or @xmath102 itself is related by another source node (line
[al:aj]), and then makes @xmath103 an downward mapping and @xmath104
an upward mapping (lines
[al:ah]-[al:ao]). The last part (lines [al:as]-[al:av]) checks
whether both root nodes are related by an upward mapping, because a \SimTran\
translation pattern must satisfy this condition. If they are not related, then
it relates them.

For instance, in the case of \figref{fig:mapxmp}, dotted arrows
gives the following lexical mappings: 
\begin{tabbing}
xxxxx\=\+\kill
\ML: \{(shiru,know),(kirei,beautiful),(kanojo,she),(hitomi,eyes)\} 
\end{tabbing}
In this example, head switching takes place for the node pairs
(kirei,beautiful) and (hitomi,eyes), and these nodes hold true for lines
[al:ad]-[al:ag]. Therefore, (kirei,have) is found as an upward
mapping. As a result, the following upward and downward mappings are obtained:
\begin{tabbing}
xxxxx\=\+\kill
\Mup: \{(shiru,know),(kirei,have),(kanojo,she),(hitomi,eyes)\} \\
\Mdw: \{(shiru,know),(kirei,beautiful),(kanojo,she),(hitomi,eyes)\} 
\end{tabbing}

For the sake of structural mapping, phrasal correspondence can be easily
obtained by exploring around two distinct nodes projected by upward and
downward mappings from one source node. For instance, in \figref{fig:mapxmp},
the source node ``kirei'' is related to the target node ``have'' by upward
mapping, and is related to the target node ``beautiful'' by downward mapping. 
The minimum connected subgraph containing ``have'' and ``beautiful'' has
another node ``eyes'' which is also related from the source node ``hitomi.'' 
This gives the phrasal correspondence between the minimum subgraph including
``kirei'' and ``hitomi'' and the minimum subgraph including ``have,''
``eyes,'' and ``beautiful'' (both are surrounded by dotted lines).
The algorithm for finding phrasal correspondence is shown in
\figref{alg:phrasalmap}. In this algorithm, the function 
@xmath105\Mup,\Mdw,@xmath106 returns a set of nodes related from @xmath107 by \Mup\
or \Mdw. Lines [al:cd]-[al:ci] find a set of source nodes and a set
of target nodes such that mappings whose sources are those source nodes and
mappings whose destinations are those target nodes are identical. These
phrasal correspondences are limited in that they cannot include extra elements
in the target part, that is, elements that are not related from any elements
in the source part.  This issue will be dealt with in the next section.

\begin{figure*}
\raggedright
\small
{\bf procedure} phrasal-correspondence(\Ds,\Dt,\Mup,\Mdw) \bfbegin\ \\
\hspace{5mm} @xmath108 \aline{al:ca}\\
\hspace{5mm} \bffor\ each pair of a node @xmath109 and its child node @xmath110 of \Ds \bfbegin\ \aline{al:cb} \\
\hspace{10mm} @xmath111 \aline{al:cc}\\
\hspace{10mm} \bfif\ there is any element @xmath112 in
@xmath113 such that @xmath114 \bfthen\ continue \aline{al:ccc}\\
\hspace{10mm} {\bf loop begin} \aline{al:cd}\\
\hspace{15mm} @xmath115\Mup,\Mdw,@xmath116 \aline{al:ce}\\
\hspace{15mm} \bfif\ there is a \Ds\ node @xmath117 such that @xmath118 and
@xmath119\Mup,\Mdw,@xmath120 \aline{al:cf}\\
\hspace{20mm} \bfthen\ @xmath121 \aline{al:cg}\\
\hspace{20mm} \bfelse\ break \aline{al:ch}\\
\hspace{10mm} \bfend \aline{al:ci}\\
\hspace{10mm} add @xmath122 to @xmath123 \aline{al:cl}\\
\hspace{5mm} \bfend \aline{al:cm}\\
\hspace{5mm} return @xmath124 \aline{al:cn} \\
\bfend\
\normalsize
\caption{Algorithm for finding phrasal correspondence}

\end{figure*}

\section{Pattern Extractor}


This section describes how to find new translation patterns.  

The basic idea for finding new translation patterns is to check whether each
translation pattern used in MT is the same as the corresponding correct
pattern.  If it is not the same, then it is a part of a new translation
pattern. Therefore, we must construct a translation pattern corresponding to
the correct translation for each translation pattern used in MT. Since the
source dependency structure is the same, we must identify the target part in
\Dc\ corresponding to a translation pattern.

Given a set of nodes in \Ds, there is a set of nodes in \Dt\ or \Dc\
projected by \Mt\ or \Mc\ from those \Ds\ nodes.  We call the minimum 
connected subgraph containing all of these projected nodes in a target
structure a {\em projected subgraph}. In \figref{fig:prjsg},
the projected nodes of S1 and S2 are T3 and T4, respectively.
Since the minimum connected subgraph including T3 and T4 must also
include T2, a projected subgraph of \Ds\ becomes a subgraph consisting of
\{T2, T3, T4\}.
\begin{figure*}
\begin{center}
\unitlength 1mm
\begin{picture}(100,50)
\put(5,5){\framebox(90,40){prjsubgraph.ps}}
\end{picture}
\end{center}
\caption{Example of projected subgraph}

\end{figure*}
Hence, for a translation pattern @xmath125 used in MT, its corresponding translation
pattern in the correct translation consists of the source part of @xmath126, a
projected subgraph of @xmath127 on \Dc, and mappings between them (a subset of \Mc).

Next, the equality of the two translation patterns must be checked.
Given two translation patterns @xmath128 and 
@xmath129, @xmath130 is identical to @xmath131 if
and only if the following conditions are all satisfied:
\begin{enumerate}
\item @xmath132 and @xmath133 are structurally identical.
\item @xmath134 and @xmath135 are identical. 
\item For each node @xmath136 of @xmath137 and its corresponding node @xmath138 in @xmath139,
let the node projected by @xmath140 be @xmath141, let the node projected
by @xmath142 be @xmath143, let the node projected by
@xmath144 be @xmath145, and let the node projected by
@xmath146 be @xmath147; then @xmath148 subsumes @xmath149 and @xmath150
subsumes @xmath151.
\end{enumerate}

An algorithm for finding translation patterns is shown in \figref{alg:trapat}. 
Lines [al:mba]-[al:mbz] check, for each translation pattern @xmath152 used
to make \Dt, whether the corresponding translation pattern @xmath153 is the
same as @xmath154 or not.  If @xmath155 is not identical to @xmath156, then it is
stored in the new translation pattern list; otherwise, it is stored in the
same translation pattern list. After that, the translation patterns in the new translation pattern
list are merged if they share any common nodes (lines
[al:mca]-[al:mcz]). Further, for each subgraph @xmath157 of \Dc\ that is
included in neither the new translation pattern list nor the same translation
pattern list, the minimum subgraph @xmath158 such that @xmath159
contains both @xmath160 and any leaf node of @xmath161 in the new translation pattern
list is found, and @xmath162 is merged with @xmath163 (lines
[al:mea]-[al:mez]).

\begin{figure*}
\raggedright
\small
{\bf procedure} find-trnpat(\Ds,\Dt,\Dc,\Mt,\Mc,@xmath164) {\bf begin}\\
\hspace{5mm} @xmath165, @xmath166 \aline{al:maa}\\
\hspace{5mm} \bffor\ each pattern @xmath167 in @xmath168 \bfbegin \aline{al:mba}\\
\hspace{10mm} @xmath169 source part of @xmath170 \aline{al:mbb} \\
\hspace{10mm} @xmath171 target part of @xmath172 \aline{al:mbc}\\
\hspace{10mm} @xmath173 mappings between @xmath174 and @xmath175 \aline{al:mbcc}\\
\hspace{10mm} @xmath176 projected subgraph of @xmath177 in \Dc\
\aline{al:mbd}\\
\hspace{10mm} @xmath178 mappings between @xmath179 and @xmath180
\aline{al:mbdd}\\
\hspace{10mm} \bfif\ identical@xmath181 \bfthen\ add @xmath182 to @xmath183
\bfelse\ add @xmath184 to @xmath185 \aline{al:mbe}\\ 
\hspace{5mm} \bfend \aline{al:mbz}\\
\hspace{5mm} \bffor\ each node @xmath186 in \Ds\ \bfbegin \aline{al:mca}\\
\hspace{10mm} @xmath187 \aline{al:mcb}\\
\hspace{10mm} remove @xmath188 from @xmath189 \aline{al:mcc}\\
\hspace{10mm} @xmath190 merge(@xmath191) \aline{al:mcd}\\
\hspace{10mm} add @xmath192 to @xmath193 \aline{al:mce}\\
\hspace{5mm} \bfend \aline{al:mcz}\\
\hspace{5mm} @xmath194 merge(all @xmath195s in @xmath196 and @xmath197) \aline{al:md}\\
\hspace{5mm} \bffor\ each disconnected subgraph @xmath198 in @xmath199
\bfbegin \aline{al:mea}\\
\hspace{10mm} @xmath200 minimum subgraph of @xmath201 that
contains @xmath202 and any leaf node @xmath203 of @xmath204 in @xmath205 \aline{al:meb} \\
\hspace{10mm} remove @xmath206 from @xmath207 \aline{al:mec}\\
\hspace{10mm} add @xmath208merge@xmath209 to @xmath210 \aline{al:med}\\
\hspace{5mm} \bfend \aline{al:mez}\\
\hspace{5mm} return @xmath211 \aline{al:mezz}\\
\bfend
\caption{Algorithm for finding translation patterns}

\end{figure*}

\section{Example}

 
This section gives an example of finding new translation patterns
by using this system. 

In \figref{fig:xmp2a}, \Ds\ is a dependency structure of the following 
Japanese sentence: \\

\begin{tabbing}{p{20mm}l}
xxx \=\+\kill
watashi no seibutugaku no tishiki ha hinjyakuda \\
(I have little knowledge of biology.) 
\end{tabbing} 

and \Dt\ is a dependency structure produced by @xmath212 as an output by using
translation patterns @xmath213, @xmath214, and @xmath215.  Dotted lines denote mappings
between two structures.  For convenience, an unmarked dotted line is
equivalent to a line marked \ML,
\Mup, and \Mdw.

\begin{figure*}
\begin{center}
\unitlength 1mm
\begin{picture}(100,70)
\put(5,5){\framebox(90,60){xmp2a.ps}}
\end{picture}
\end{center}
\caption{Translation patterns and dependency structures of input and
translation by @xmath216}

\end{figure*}
\begin{figure*}
\begin{center}
\unitlength 1mm
\begin{picture}(100,140)
\put(5,5){\framebox(90,130){xmp2b.ps}}
\end{picture}
\end{center}
\caption{Translation patterns and dependency structures of input and correct translation}

\end{figure*}

Suppose that \Dc\ in \figref{fig:xmp2b} is a correct English translation of 
\Ds. Then the first step is to find mappings from \Ds\ to
\Dc.  By using a procedure described in \secref{sec:mapping}, the mappings
expressed by dotted lines in \figref{fig:xmp2b} are obtained. In these
mappings, note that the Japanese word ``hinjyaku'' (``poor'') is related to
the English word ``have'' even though it is not a translation word. This
relationship can be established in the last part of the mapping algorithm.

The next step is to find structural differences between the translation
patterns for @xmath217 and for @xmath218. The
translation patterns corresponding to @xmath219, @xmath220, and @xmath221 are shown as
@xmath222, @xmath223, and @xmath224 in \figref{fig:xmp2b}.
Comparison of these corresponding translation patterns shows that
@xmath225 and @xmath226 are different, and they are stored in the
new translation pattern list. \Dc\ contains a portion, ``little (mod),'' that
is not covered by the translation patterns @xmath227, @xmath228, and
@xmath229.  This portion is attached to @xmath230, and becomes
@xmath231 in \figref{fig:xmp2b}.  Finally, a new translation
pattern @xmath232 is obtained by merging @xmath233 and
@xmath234 in the new translation pattern list.
 
\begin{figure*}[tb]
\begin{center}
\unitlength 1mm
\begin{picture}(100,130)
\put(5,5){\framebox(90,120){screensave.ps}}
\end{picture}
\end{center}
\caption{Screen image of TranPet}

\end{figure*}
\begin{figure*}[tb]
\begin{center}
\unitlength 1mm
\begin{picture}(100,100)
\put(5,5){\framebox(90,90){screensave2.ps}}
\end{picture}
\end{center}
\caption{New translation pattern found by TranPet}
 
\end{figure*}

The actual screen images of \TranPet\ about this example are shown in
\figref{fig:screen1} and \figref{fig:screen2}. In \figref{fig:screen1}, the
source dependency structure is presented in the upper left window, and the
target dependency structure produced by \SimTran\ is presented in the lower
left window. The upper right window contains a correct translation in English
inputed by a user. By pushing {\em analysis} button, its parsed
dependency structure appears in the middle right window. {\em Find
Mapping} button executes the Mapper, and the mappings found by the Mapper
appears in the lower right window. Finally, {\em Find Pattern} button
executes the Pattern Extractor, and the new translation pattern is displayed
in another window as shown in
\figref{fig:screen2}. 

\section{Discussion}


Strong recent interest in corpus-based processing has produced some results
concerning the extraction of relevant information from bilingual corpora. 
There has been some research
@xcite on finding
correspondences between translation pair sentences, and extracting translation
patterns from them.  The method proposed in this paper differs from previous
ones in that it extracts new relevant translation patterns (which are not
contained in the current translation pattern base) by comparing the result
output by a translation system with a correct translation.

If we consider human processes for acquiring translation patterns, we can
categorize them into two types: finding translation patterns by viewing many
similar translations, and finding translation patterns by comparing a correct
translation and a wrong translation.  An example of the former is that, given
many translation instances of the Japanese word ``toru'' (``take''), one can
obtain some translation patterns (or case frames) that are relevant for
translating sentences containing ``toru.'' An example of the latter type has
been given in this paper. The above-mentioned studies
@xcite deal with the former type of process, whereas the
method proposed in this paper deals with the latter.  The former type of
method is necessary for creating a new translation pattern base from
scratch; on the other hand, the latter type is effective for
enhancing an existing translation pattern base.

Previous studies @xcite proposed a method for
extracting not only one-to-one correspondence but also phrasal correspondence.
As we mentioned in previous sections, our system also obtains structural
correspondence; limited phrasal correspondence can be obtained in the mapping
phase, and more global phrasal correspondence can be obtained by finding new
translation patterns.

Some may feel that the proposed method is peculiar to \SimTran. Admittedly, a
method for finding structural mappings is unique to \SimTran, but a method for
finding new relevant translation patterns is applicable to any type of
example-based transfer system, if lexical mapping is substituted for
structural mappings in the descriptions above.

Since this system is an interactive tool to assist in creating translation
patterns, it is not easy to evaluate. We have actually been using this system for
several months to create \SimTran\ translation patterns. The experience
indicates that, if the mapping is correct, the result of the system can mostly
be used as a translation pattern after a little editing.  Most errors in the
mapping phase are caused by (1) a word used in a correct translation sentence
not being registered in the dictionary as a translation for its source word,
and (2) a single source word, such as a compound noun phrase, being translated
as several target words.  Further, creating a translation pattern by using
this system takes half as much time as creating one from scratch.
Therefore, this system improves the productivity of pattern creation largely.

\section{Conclusion}


In this paper, I have described a system called \TranPet\ that compares a
wrong translation and a correct translation in order to extract relevant
translation patterns that can be added to a current translation pattern base. 
To find mappings between two parsed structures, I proposed a method for
finding not only one-to-one correspondence but also structural mappings
employed in \SimTran. To find new translation patterns, I proposed a method
for finding translation patterns from the differences between translation
patterns used in a wrong translation and those used in a correct translation. 
This method is useful for extending or enhancing a current translation pattern
base efficiently.

This system has been actually used for creating \SimTran\ translation patterns
in {\bf JETS} Japanese-to-English MT system. As for the future work,
in addition to improving parsers, we must enhance a part finding mappings
to reduce user interaction.


\bibliography{jnlp94}
\end{document}