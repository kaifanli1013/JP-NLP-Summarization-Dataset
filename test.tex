\title{支配従属構造照合による文と名詞句の前方照応解析}
\author{吉見 毅彦\affiref{SHARP/KOBE} \and
Jiri Jelinek\affiref{EXSHARP}}

\headauthor{吉見 毅彦・Jiri Jelinek}
\headtitle{支配従属構造照合による文と名詞句の前方照応解析}

\affilabel{SHARP/KOBE}{シャープ(株) 情報商品開発研究所 / 神戸大学 大学
院自然科学研究科}
{Information Systems Product Development Laboratories, SHARP Corp. /
Graduate School of Science and Technology, Kobe University}
\affilabel{EXSHARP}{元 シャープ(株) 情報商品開発研究所}
{Formerly, Information Systems Product Development Laboratories, SHARP
Corp.}

\jabstract{本稿では,文とその後方に位置する名詞句との照応を,複雑な知
識や処理機構を用いず,表層的な情報を用いた簡単な処理によって解析する
方法を提案する.
文と名詞句の構文構造を支配従属構造で表現し,それらの構造照合を行ない,
照合がとれた場合,照応が成立するとみなす.
構造照合に用いる規則は,文が名詞句に縮約されるときに観察される現象のう
ち,主に,用連助詞から体連助詞への変化,情報伝達に必須でない語の削除に
着目して定義する.
このような簡単な処理によって前方照応がどの程度正しく捉えられるかを検証
するための実験を,サ変動詞が主要部である文と,そのサ変動詞の語幹が主要
部である名詞句の組を対象として行なった.
実験では,新聞記事から抽出した178組のうち133組(74.7\%)について,本手法
による判定と人間による判定が一致した.
また,構造照合で類似性が最も高いと判断された支配従属構造の組を優先解釈
として出力することによって,入力の時点で一組当たり平均3.4通り存在した
曖昧性が1.8通りへ絞り込まれた.}

\jkeywords{照応,支配従属構造,構造照合,曖昧性解消}

\etitle{Anaphora Resolution of \\
Sentences and Noun Phrases \\
by Matching Dependency Structures}
\eauthor{Takehiko Yoshimi\affiref{SHARP/KOBE} \and 
Jiri Jelinek\affiref{EXSHARP}}

\eabstract{We propose a simple method of analysing correferentiality
between sentences and later occurring noun phrases. 
Our method uses surface information and requires no complex data or
processing mechanism.
We represent a sentence and a later occurring noun phrase as dependency
structures, and examine whether the two structures are matched.
Where a matching between them can be established, we assume that the
two are correferential.
The rules for establishing structural matching are part of the
paradigm of theme packing, namely the predictable changes of adverbal
particles into adnominal particles and the disappearance of some
non-essential information.
In order to ascertain to what degree anaphora can be correctly traced
by such simple processing, we have carried out an experiment centred
upon sentences governed by a verb of the SAHEN category and later
occurring noun phrases in which the head noun is formally identical
with the invariable part of the SAHEN verb.
Of the 178 pairs of such sentences and noun phrases selected
from newspaper articles, 133 pairs (74.7\%) were correctly identified
as correferential or otherwise, in accordance with human judgement.
Furthermore, as a side effect, the number of dependency
structures to be considered can be 
reduced by selecting only the pairs of dependency structures with the 
best affinity through structural matching.
By this method the average 3.4-fold structural ambiguity was reduced to
average 1.8-fold.}

\ekeywords{Correference, Dependency, Structural Matching, Disambiguation}

\begin{document}
\maketitle

\section{はじめに}
\label{sec:introduction}

適格なテキストでは,通常,テキストを構成する要素の間に適切な頻度で照応
が認められる.
この照応を捉えることによって,テキスト構成要素の解釈の良さへの裏付けや,
解釈の曖昧性を解消するための手がかりが得られることが多い.
例えば,次のテキスト[TEXT:shiji] の読み手は,「新自由クラブは,奈良
県知事選で自民党推薦の奥田氏を支持する」で触れた事象に,「知事選での奥
田氏支持」が再び言及していると解釈するだろう.
\begin{TEXT}
\text \underline{新自由クラブは,奈良県知事選で自民党推薦の奥田氏を支
持する}方針をようやく固めた.
\underline{知事選での奥田氏支持}に強く反対する有力議員も多く,決定が今
日までずれ込んでいた.\label{TEXT:shiji}
\end{TEXT}
この照応解釈は,「奈良県知事選で」が「支持する」と「固めた」のどちらに
従属するかが決定されていない場合には,この曖昧性を解消するための手がか
りとなり,何らかの選好に基づいて「支持する」に従属する解釈の方が既に優
先されている場合には,この解釈の良さを裏付ける.

このようなことから,これまでに,前方照応を捉えるための制約(拘束的条件)
と選好(優先的条件)がText-Wide Grammar~@xciteなどで提案されて
いる.
Text-Wide Grammarによれば,テキスト[TEXT:shiji] でこの照応解釈が成
立するのは,「新自由クラブは,奈良県知事選で自民党推薦の奥田氏を支持す
る」を@xmath0,「知事選での奥田氏支持」を@xmath1としたとき,これらが次の三つの
制約を満たすからである.
\smallskip
\begin{LIST}
\item[\bf 構文制約] @xmath2は,ある構文構造上で@xmath3の後方に位置する
.
\item[\bf 縮約制約] @xmath4は,@xmath5を縮約した言語形式である.
\item[\bf 意味制約] @xmath6の意味は,@xmath7の意味に包含される.
\end{LIST}
\smallskip
あるテキスト構成要素@xmath8で触れた事象に他の要素@xmath9が再言及しているかどう
かを決定するためには,@xmath10と@xmath11がこれらの制約を満たすかどうかを判定する
ための知識と機構を計算機上に実装すればよい.
実際,構文制約と縮約制約については,実装できるように既に定式化されてい
る.
これに対して,意味制約が満たされるかどうかを具体的にどのようにして判定
するかは,今後の課題として残されている.

意味制約が満たされるかどうかを厳密に判定することは,容易ではない.
厳密な判定を下すためには,@xmath12と@xmath13の両方またはいずれか一方が文や句であ
る場合,その構文構造とそれを構成する辞書見出し語の意味に基づいて全体の
意味を合成する必要がある.
テキストの対象分野を限定しない機械翻訳などにおいて,このような意味合成
を実現するためには,膨大な量の知識や複雑な機構を構築することが必要とな
るが,近い将来の実現は期待しがたい. 

本稿では,近い将来の実用を目指して,構築が困難な知識や機構を必要とする
意味合成による意味制約充足性の判定を,表層的な情報を用いた簡単な構造照
合による判定で近似する方法を提案する.
基本的な考え方は,構文制約と縮約制約を満たす@xmath14と@xmath15について,それぞ
れの構\\文構造を支配従属構造で表し,それらの構造照合を行ない,照合がとれ
た場合,@xmath16の意味が@xmath17の意味を包含するとみなすというものである.
もちろん,単純な構造照合で意味合成が完全に代用できるわけではないが,本
研究では,日英機械翻訳への応用を前提として,簡単な処理によって前方照応
がどの程度正しく捉えられるかを検証することを目的とする.
以降,本稿の対象を,サ変動詞が主要部である文(以降,サ変動詞文と呼ぶ)を
@xmath18とし,そのサ変動詞の語幹が主要部であり@xmath19の後方に位置する名詞句(サ
変名詞句)を@xmath20とした場合
に限定する.

これまでに,性質の異なる曖昧性がある二つの構文構造を照合することによっ
て互いの曖昧性を打ち消す方法に関する研究が行なわれ,その有効性が報告さ
れている@xcite.
本稿の対象であるサ変動詞文とサ変名詞句にも互いに性質の異なる曖昧性があ
るので,構造照合を行ない,類似性が高い支配従属構造を優先することによっ
て,サ変動詞文とサ変名詞句の両方または一方の曖昧性が解消される.
例えば,サ変名詞句「奥田氏支持」から得られる情報だけでは「奥田氏」と
「支持」の支配従属関係を一意に決定することは難しいが,テキスト
[TEXT:shiji] では,サ変動詞文「奥田氏を支持する」との構造照合によっ
て,サ変名詞句を構成する要素間の支配従属関係が定まる.

このように,サ変名詞句の曖昧性解消に,サ変名詞句の外部から得られる情報
を参照することは有用である.
一方,複合名詞の内部から得られる情報に基づく複合名詞の解析法も提案され
ている@xcite.
複合名詞の主要部がサ変名詞である場合,これら二つの方法を併用することに
よって,より高い解析精度の達成が期待できる.

[sec:depredrules] 節では,サ変動詞文とサ変名詞句の支配従属構造を照
合するための規則を記述する.
[sec:matching] 節では,構造照合規則に従って照応が成立するかどうかを
判定する手順について述べ,処理例を挙げる.
[sec:experiment] 節では,新聞記事から抽出したサ変動詞文とサ変名詞句
の組を対象として行なった実験結果を示し,照応が正しく捉えられなかった例
についてその原因を分析する.

\section{サ変動詞文とサ変名詞句の構造照合規則}
\label{sec:depredrules}

サ変動詞文がサ変名詞句に縮約されるときに観察される現象のうち次の現象に
着目して,サ変動詞文とその後方に位置するサ変名詞句の間で照応が成立する
ときに従うべき規則を定める.
\begin{enumerate}
\item  サ変動詞とその従属語を関係付ける助詞(以降,用連助詞と呼ぶ)は,
その支配従属関係を保ったまま,サ変名詞とその従属語を関係付ける助詞(体
連助詞)に変化する.
\item サ変動詞文でのサ変動詞の態の区別は,サ変名詞句では制限されるか,
あるいは行なわれなくなる.
\item サ変名詞句では,サ変動詞の従属語のうち,情報伝達に必須である語のみ
が,サ変動詞文での出現順序に因われない順序で表現される.
\end{enumerate}

\subsection{用連助詞から体連助詞への変化}

サ変動詞文がサ変名詞句に縮約されるとき,サ変動詞文におけるサ変動詞とそ
の従属語の支配従属関係は,サ変名詞句におけるサ変名詞とその従属語の支配
従属関係として保存される.
ただし,サ変動詞と従属語を関係付ける用連助詞は,規範に従って,サ変名詞
と従属語を関係付ける体連助詞に置き換えられるか,あるいは表現されなくな
る.
例えば,サ変動詞文「奥田氏を支持する」における用連助詞「を」が体連助詞
「の」またはゼロ形態素に変化することによって,それぞれ,サ変名詞句「奥
田氏の支持」または「奥田氏支持」に縮約される.
用連助詞から体連助詞への変化は,一般に,表[tab:youren2tairen] に示
す対応に従う.
動詞あるいは動詞型に固有の変化,例えば,「イラク説得の成功を期待する」
から「成功への期待」への縮約に見られる用連助詞「を」から体連助詞「への」
への変化は,表[tab:youren2tairen] の対応とは別に,動詞あるいは動詞
型毎に記述する.
\begin{RULE}
\rule 用連助詞は,それが表す支配従属関係と深層的に同じ支配従属関係
\footnotemark
を表す体連助詞に変化する.
その変化は,動詞固有の対応が記述されていれば,それに従い,さもなければ
表[tab:youren2tairen] に従う.\label{RULE:youren2tairen}
\end{RULE}
\footnotetext{本稿で行なっている支配従属関係の区別は,比較的浅い深層度
での区別である.
例えば,助詞「が」と「に」が表す支配従属関係の一つは,より深い層では共
に``動作主''になりえるが,ここではSubject,Agentという別の関係としてい
る.
その理由は,本稿での区別は日英機械翻訳での利用を前提としたものであり,
この限りにおいて,表層表現をより深い層へ写像する必要はないからである.}


\subsection{態の区別の制限}

サ変動詞文では,サ変動詞に接続する助動詞によって,能動態,受動態,使役
態,間接受動態,可能態と,これらの組合せ(受動使役態など)が区別される.
これに対して,サ変名詞句では,態の区別が制限されるか,あるいは行なわれ
なくなる.
例えば,次のテキスト[TEXT:sabetsu] では,サ変動詞文「日本ではアジア
人が差別される」が受動態であったことは,サ変名詞句「足元にいるアジア人
に対する差別」では表現されていない.
\begin{TEXT}
\text \underline{日本ではアジア人が差別される}のは当然だという考え方が
強い.
日本にとって国際化とは,\underline{足元にいるアジア人に対する差別}を撤
廃して,一緒に手を組んで生きてゆくことだと思う.\label{TEXT:sabetsu}
\end{TEXT}

表[tab:youren2tairen] の対応は,サ変動詞の態が能動である場合の対応
である.
従って,サ変動詞が能動態でない場合,用連助詞から体連助詞への変化が規則
[RULE:youren2tairen] に従うかどうかを調べる前に,サ変動詞の態を能動
態に戻しておく必要がある.
\begin{RULE}
\rule サ変動詞の態が能動でなければ,サ変動詞とその従属語との支配従属関
係は表[tab:voice] に示す対応に従って変化する.\label{RULE:voice}
\end{RULE}


\subsection{情報伝達に必須でない従属語の削除}

サ変動詞の従属語のうち,情報伝達に必須でない語は,サ変名詞句では表現され
なくなる.
また,削除されずに残ったサ変名詞の従属語の出現順序は,サ変動詞文におけ
るサ変動詞の従属語の出現順序に一致するとは限らない.
例えば,サ変動詞文「新自由クラブは,奈良県知事選で自民党推薦の奥田氏を
支持する」は,
1)すべての語を元の出現順序のままで表現した「新自由クラブによる奈良県知
事選での自民党推薦の奥田氏の支持」から,
2)「新自由クラブ」と「自民党推薦」が削除され,「奈良県知事選」と「奥田
氏」の出現順序が交替した「奥田氏に対する奈良県知事選での支持」などを経
て,
3)サ変名詞以外の語をすべて削除した「支持」に至るまで,
様々なサ変名詞句に縮約されうる.

「奈良県知事選で支持する」から「知事選での支持」への縮約における「奈良
県知事選」から「知事選」への変化に見られるように,情報伝達に必要でない
従属語は,語全体ではなくその部分が削除されることがある.
従属語(名詞)は,@xmath21という構成\footnote{記号 @xmath22
は0回以上の繰り返し,@xmath23 は1回以上の繰り返しを意味する.}をしており,語
の主要部は最後尾の語基である.
従属語の一部が削除されるとき,まず削除されるのは主要部以外の部分であり,
語の主要部である最後尾の語基は最後まで削除されずに残る場合が多いので,
次のような規則をおく.
\begin{RULE}
\rule サ変名詞の従属語は,サ変動詞のいずれかの従属語から接尾語を除いた
部分に後方文字列一致する.
\label{RULE:stringmatch} 
\end{RULE}


\section{構造照合による意味合成の近似}
\label{sec:matching}

[sec:depredrules] 節で述べた構造照合規則を用いたサ変動詞文とサ変名
詞句の照応判定は,図[fig:algorithm] に示す手順に従って行なう.
この手順に従う処理によって照応が成立すると判定される例,成立しないと
判定される例を示す.


照応が成立すると判定される例として,[sec:introduction] 節のテキスト
[TEXT:shiji] を処理する過程を追う.
サ変動詞文を含む文とサ変名詞句を含む文に対して形態素,構文,意味的共起
解析を行ない,可能な支配従属構造のうち,形態素,構文,意味的共起に関す
る選好による総合評価点が最も高い構造を入力とすると,テキスト
[TEXT:shiji] の場合,図[fig:depstruct] に示すように,サ変動詞文
の支配従属構造が3通り,サ変名詞句の支配従属構造が2通り抽出される
.
抽出されたサ変名詞句の支配従属構造を@xmath24とする.
また,抽出されたサ変動詞は能動態であり,規則[RULE:voice] に従って支
配従属関係を書\\き換える必要はないので,抽出されたサ変動詞文の支配従属構
造をそのまま@xmath25とする.
@xmath26は,「支持する」が「新自由クラブは」と「奈良県知事選で」と「奥田
氏を」を支配する解釈,\\
@xmath27は,「支持する」が「奈良県知事選で」と「奥田氏を」を支配し,第一
文の主動詞「固めた」が「新自由クラブは」を支配する解釈,
@xmath28は,「支持する」が「奥田氏を」を支配し,「固めた」が「新自由クラ
ブは」と「奈良県知事選で」を支配する解釈である.
また,@xmath29は,「支持」が「知\\事選での」と「奥田氏を」を支配する解釈,
@xmath30は,「支持」が「奥田氏」を支配し,「奥田氏」が「知事選での」を支
配する解釈である.


ステップ[ALGO:matching] で,@xmath31と@xmath32の組合せから成る
6通りの対について,それぞれマッチング問題を解く.
頂点の支配従属関係が集合として表されている場合,支配従属関係の集合の
交わりが空でないときに限り二つの頂点が辺で結べることにすると,対[@xmath33]についての処理では,図[fig:matching1] に示すように,
@xmath34の二つの頂点@xmath35,
@xmath36を
@xmath37の頂点
@xmath38と
結び,@xmath39の頂点@xmath40を
@xmath41の頂点@xmath42と結んだ二部グ
ラフが構成できる.
この二部グラフでは,実線の辺で結ばれている頂点を組にすることによって,
評価点2点の最大マッチングが得られる.
なお,辺で結ばれている頂点を組にしたとき,頂点の支配従属関係は元の集合
から集合の交わりに変化する.
他の対についても同様に処理すると,対[@xmath43]と[@xmath44]から得ら
れる最大マッチングが全体で最も評価点が高いマッチングとなる.


全体で最も評価点が高いマッチングが得られる支配従属構造の対を優先させる
ことによって,サ変動詞文とサ変名詞句の両方または一方の曖昧性を絞り込む
ことができる.
@xmath45と@xmath46の構造照合の結果,図[fig:matching1] の最大マッチングが
得られることから,@xmath47における「奥田氏」と「支\\持」の支配従属関係が
Objectに定まる.
もう一つの対[@xmath48]についての処理でも,@xmath49における「奥田氏」と
「支持」の支配従属関係が一意に定まる.
従って,テキスト[TEXT:shiji] の場合,全体で\\@xmath50通りの
可能性を2通りに絞り込めたことになる.

次のテキスト[TEXT:teishi] を対象とした処理では,サ変動詞文「大統領
がリトアニアに対する強硬手段を停止する」とサ変名詞句「ECによる援助の停
止」の間に照応は成立しないと判定される.
\begin{TEXT}
\text ソ連軍による弾圧は,\underline{大統領がリトアニアに対する強硬手
段を停止する}との方針を明らかにした直後に起きた.
これに対しベルギー外務省筋は\underline{ECによる援助の停止}もあり得る
と語っている.\label{TEXT:teishi}
\end{TEXT}
サ変動詞文の支配従属構造としては,「停止する」が「大統領が」を支配する
場合とそうでない場合の2通りが可能であり,サ変名詞句の支配従属構造とし
ては,「停止」が「ECによる」を支配する場合とそうでない場合の2通りが可
能であるので,支配従属構造の各対について規則[RULE:youren2tairen]
に従う頂点を辺で結ぶと,図[fig:matching2] に示すような4通りの二部グ
ラフが構成できる.
しかし,いずれの二部グラフにおいても規則[RULE:stringmatch] に従う頂
点が存在しないので,組が構成できない. 
従って,サ変動詞文とサ変名詞句の間に照応は成立しないと判定される.


\section{実験と考察}
\label{sec:experiment}

支配従属構造照合による意味制約充足性の判定法の有効性を検証するために,
新聞記事データ@xciteからサ変動詞文とサ変名詞句を含む100記事を
無作為に抽出し,各記事に含まれるサ変動詞文とサ変名詞句の組のうち,両
者の物理的な距離が最も近いもの178組を対象として実験を行なった.
サ変動詞文を含む文とサ変名詞句を含む文に対して形態素,構文,意味的共起
解析を行ない,可能な支配従属構造のうち,形態素,構文,意味的共起に関す
る選好による総合評価点が最も高い構造を入力とした.
なお,処理対象のサ変動詞文とサ変名詞句は,サ変動詞の語幹がサ変名詞句の
主要部であるため照応が成立する可能性が高いと考えられるが,人間に照応が
認められるのは178組中95組(53.4\%)であった.

\subsection{照応判定成功率}

本手法による照応判定と人間による判定を比較した結果を表[tab:result]
に示す.
本手法と人間による判定が,照応成立という解釈で一致したサ変動詞文とサ変
名詞句は77組(43.2\%),不成立という解釈で一致したのは56組(31.5\%)であっ
た. 
この結果,本手法による判定成功率は74.7\%となる.
また,判定を誤った組のうち,人間には照応が認められるものを照応不成立と
判定した組は18組(10.1\%),その逆は27組(15.2\%)であった.



照応不成立と誤判定した18組について,その原因を分析した結果を表
[tab:fail] に示す.
最も多かった原因は,サ変名詞の従属語としてサ変動詞の従属語あるいはその
一部が用いられず,その類義語などの別の語が用いられていることであった.
\begin{FAIL}
\fail  ただ,\underline{歴訪を中止し}て湾岸危機の平和的解決に努める,
といっても,具体的な手立ては乏しい.
(中略)\ 
首相は11日朝,\underline{ASEAN訪問中止}の理由を記者団から聞かれ,「米
国とイラクの9日の外相会談が不調に終わって総合的に判断して決めた」.
\label{FAIL:chuushi}
\end{FAIL}
失敗例[FAIL:chuushi] では,「歴訪」と「訪問」の文字列照合に失敗する
ので,照応は成立しないと誤判定された.

サ変動詞文では明示されていない語がサ変名詞句で初めて表現されているため,
判定を誤った例を示す.
\begin{FAIL}
\fail しかし,期限切れ前に締結される今回の新特別協定によって,これまで
の諸手当に加え,日本人従業員の基本給と光熱水費を91年度から段階的に肩代
わりし,\\
\underline{新中期防衛力整備計画の最終年度の95年度には,その全額を負担でき
る}仕組みになっている.
(中略)\ 
必要経費は約2200億円で,\underline{日本側負担}分は,95年度に在\\日駐留経
費全体の約半分に達する見通しだ.\label{FAIL:futan}
\end{FAIL}
失敗例[FAIL:futan] の場合,明示されていないサ変動詞文の主語が「日本
側」であることがすでに推論されているならば,本手法をそのまま適用するこ
とによって照応が正しく捉えられ,推論の正しさが裏付けられる.
しかし,明示されていない主語が「日本側」であるとの推論をサ変動詞文を含
む文から得られる情報だけに基づいて行なうことは容易ではない.
むしろ,本手法による判定から得られる情報を曖昧性絞り込みの手がかりとし
て,より積極的に利用する方が容易である.
サ変動詞文で明示されていない従属語に関する情報が完全には得られておらず,
例えば,主語が明示されていないことだけが分かっており,それが具体的に何
であるかは分かっていないならば,任意の文字列と照合がとれる節点を主語と
してサ変動詞文の従属構造構造に加えておけば,本手法の判定によって曖昧性
絞り込みの手がかりが提供されるようになる.

人間には照応が認められないものを照応成立と誤判定した27組についての原因
は,すべて,サ変動詞文とサ変名詞句の間に照応が成立しないことを示す手が
かりが,サ変動詞文とサ変名詞句以外の部分にあることによるものであった.
失敗例[FAIL:koukan] では,サ変動詞文「フセイン・イラク大統領と意
見を交換する」とサ変名詞句「意見交換」の間に照応が成立しないと判定する
ためには,「フセイン国王との間で」の部分を参照しなければならない. 
\begin{FAIL}
\fail 
同事務総長は空港で報道陣に対し,「わたしはいかなる和平案も携行して
いない.\underline{フセイン・イラク大統領と意見を交換し}に行く」と手短
に答え,ジュネーブで言及したイラク軍撤退後の国連平和維持軍派遣の構想な
どには触れなかった.
この後,王宮に向かい,欧州4カ国歴訪から9日に帰国したばかりのフセイン国
王との間で,\underline{意見交換}を行った.\label{FAIL:koukan}
\end{FAIL}
特に,27組中21組については,サ変名詞句がサ変名詞だけで構成されているた
めに,照応が成立しないことを示す情報がサ変名詞句からは得られなかった.
今回の実験では,サ変動詞の語幹がサ変名詞句の主要部である場合に対象を限
定しているため,サ変名詞句がサ変名詞だけで構成されている場合,照応が成
立する可能性が高いと考え,そのように判定している.
実際,そのように判定することで,サ変名詞だけで構成されているサ変名詞句
を含む73組のうち,52組(71.2\%)について正しい判定が下されている.

本手法では,各記事に含まれるサ変動詞文とサ変名詞句の組のうち,両者の距
離が最も近いものを対象としてはいるが,サ変動詞文とサ変名詞句の距離に関
する制約を課していないので,サ変名詞句がサ変名詞だけで構成されている場
合,両者がどんなに離れていても照合成立と判定される.
距離に関するパラメータと算式を定めることは今後の課題である.

\subsection{曖昧性解消率}

本手法の判定によって照応が成立すると正しく判定された77組のサ変動詞文と
サ変名詞句について,類似性が最も高いと判断された(全体で最も高い評価点
が与えられた)支配従属構造の組を優先することによって,どの程度曖昧性が
絞り込めるかを調べた.
本手法への入力である支配従属構造から抽出されたサ変動詞文とサ変名詞句に
関係する部分の解釈の可能性は一組当たり平均3.4通り存在したが,これを構
造照合によって1.8通りへ絞り込めた.
可能な解釈の数が減少した組は18組であり,このうち17組は絞り込まれた解釈
の中に,人間によって正しいと判定される解釈が含まれていた.
また,18組中10組については解釈を一意に決定でき,そのうち9組が人間によっ
て正しいと判定される解釈であった.

\section{おわりに}

本稿では,テキストの対象分野を限定しない日英機械翻訳への応用を前提とし
て,構築が困難な知識や機構を必要とする意味合成による意味制約充足性の判
定を支配従属構造照合による判定で近似する方法を示し,実験を行なった.
本手法による照応判定を第一次近似とみなせば,比較的精度が高い結果が得ら
れた背景には,英語では,先行文脈中に現れた語をそのまま反復すること
は文体上の理由から希であり,別の語に言い換える場合が多いのに対し,
日本語では既出の語をそのまま繰り返す場合が多いことがあるものと考え
られる.
今回の実験では,処理対象をサ変動詞文とサ変名詞句に限定し,文字列情報の
みを用いて処理を行なったが,類義関係を記述した意味体系が利用可能ならば,
対象を拡げることも可能である.

\acknowledgment
新聞記事データの利用を承諾下さった朝日新聞社の関係者の方々に感謝いたし
ます.

